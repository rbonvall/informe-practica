\documentclass[12pt]{article}
\usepackage[utf8]{inputenc}
\usepackage{fullpage}
\usepackage{url}
\usepackage{palatino}
\usepackage{graphicx}

\title{Internship report}
\author{Roberto Bonvallet \\ \url{rbonvall@inf.utfsm.cl}}
\date{November 2nd, 2009}

\begin{document}
  \maketitle

  \section{Student information}
  \begin{description}
    \item[Name:] Roberto Bonvallet
    \item[USM ID:] 2173014-9
    \item[Email address:] \url{rbonvall@inf.utfsm.cl}
    \item[Phone:] no phone
  \end{description}

  \section{General description of the organization}
    The European Organization for Nuclear Research, known as CERN, is the
    world's largest particle physics laboratory, situated near Geneva,
    on the border between France and Switzerland.  CERN's main function is to
    provide the particle accelerators and other infrastructure needed for
    high-energy physics research.

    CERN operates a network of accelerators, of which the most important is the
    Large Hadron Colider (LHC), which is being built.  This accelerator will
    generate vast quantities of computer data, which CERN will stream to
    laboratories around the world for distributed processing, by making use of
    a specialized Grid computing infrastructure, the LCG.

  \section{Section description}
    I worked in the Grid Operations section of the Grid Deployment
    group, within the Information Technology (IT) Department.

    The \emph{IT Department} is responsible of fulfilling CERN's needs for information
    technology services for general-purpose, administrative, physics and
    engineering computing, and the consolidation, coordination and
    standardization of computing activities.

    The \emph{Grid Deployment group} is responsible for deploying, operating and
    supporting the LCG service, and for coordinating the grid operations and
    management activity within the Enabling Grids for E-sciencE project. 
    The LCG (LHC Computing Grid) is the distributed computing and data
    storage infrastructure supporting the LHC experiments.

    The \emph{Grid Operations section} is responsible for: operating grid-specific
    services (information providers, resource brokers, authentication and
    authorization services, etc.), and for building and operating a distributed
    Grid Operations Centre, to perform performance monitoring, problem
    identification, troubleshooting and general operational oversight.
    providing 24$\times$7 global coverage.

    The organization chart is attached as an appendix to this document.

  \section{Main tasks accomplished}
    The main tasks that were assigned to me were:
    \begin{enumerate}
        \item the design of a distributed dynamic system for the
            synchronization of firewalls among Grid sites, and
        \item the monitoring of the VOMS and VOMRS installations at the CERN
            site.
    \end{enumerate}

    \subsection{Firewall system}
    At the time of my arrival at the group, there was already an application
    being used for the synchronization of firewalls, but it was meant to
    work only in a local site, and not among sites.  This system was designed,
    implemented and deployed on the CERN site by Romain Wartel%
    \footnote{Romain Wartel: office~28~R-016, phone~74585, \url{Romain.Wartel@cern.ch}},
    who wanted to improve it and assigned this project to me.

    The main idea was to isolate all Grid components from the rest of the
    Internet, by configuring their local firewalls to accept only connections
    that come from hosts that belong to Grid sites, and to keep these
    configurations updated dinamically.

    My proposed solution is a three-component system that queries periodically
    an already existent centralized database that contains site-specific data,
    publishes it through HTTP to the local site, in order to allow clients on
    its hosts to pull it and configure the local firewall.

    The system is fully documented on
    \url{https://twiki.cern.ch/twiki/bin/view/LCG/LCGFirewallSystem}.
    
    \subsection{VOMS monitoring}
    The VOMS service handles the authentication and authorization of users when
    using the Grid.  It is a critical service that must be working most of the
    time.  However, the CERN instalation of VOMS and VOMRS (a related service)
    used to fail often, sometimes due to a well-known bug in the application%
    \footnote{\url{https://savannah.cern.ch/bugs/index.php?16843}}
    (which could not be reproduced nor fixed at that time) but also for other
    reasons.

    With the direction of Maria Dimou%
    \footnote{Maria Dimou: office 28 R-020, phone 73356, \url{Maria.Dimou@cern.ch}},
    I worked on ways to monitor the servers in order to foresee when a failure
    was going to happen and to recover and report when it actually happened.

    Part of this task involved reviewing log files and correlating unusual
    events with the failures that were recorded.

    For the implementation of the monitoring tools, I decided to use Lemon%
    \footnote{Lemon.  \url{http://cern.ch/lemon}},
    a monitoring framework developed at CERN.  After learning about Lemon,
    I was able to use existing sensors to monitor the service, and to develop my
    own sensors for more VOMS-specific monitoring.  This experience was very
    valuable, since there was no person with deep Lemon knowledge working in my
    section at that time, and the deployed sensors proved to be helpful.


  \section{Technological platform}
    The technological platform on which my work was done was the LCG Grid
    infrastructure, in particular with the Grid Security Infrastructure
    component.

    




  \section{Personal contribution}
    


  \section{Dates}
    My internship lasted from July 15th, 2006, to January 15th, 2007.
    

    %During my stay at CERN, which lasted from July 15th, 2006 to January
    %15th, 2007, I worked in the Grid Operations section of the Grid Deployment
    %group, within the LCG project.  My supervisor was Ian Neilson%
    %\footnote{Ian Neilson: office 28 R-020, phone 74929, \url{Ian.Neilson@cern.ch}}
    %and my work was related to security in the Grid.

\newpage
  %\section*{Appendices}
  \section*{Appendix: Organization chart}
    The first chart shows the organization structure of the IT Department.
    I worked in the Grid Deployment group, represented by the fourth box in the
    bottom row.

    \includegraphics[width=\textwidth]{ITorgJAN2008.pdf}

    The second chart shows the structure within the Grid Deployment group.
    I worked in the security group of the Grid Operations section.

    \begin{center}
    \includegraphics[width=0.5\textwidth]{it-gd.pdf}
    \end{center}

\end{document}

% a. Datos del alumno (Nombre, rol utfsm, e-mail, teléfono).
% b. Descripción general de la Empresa.
% c. Descripción de la unidad donde trabajó (organigrama, cargos, lugar)
% d. Las tareas principales realizadas.
% e. Plataforma tecnológica utilizada.
% f. Califique su aporte personal a la Organización.
% g. Fechas en las que realizó la práctica.


